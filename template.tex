\documentclass[10pt]{article}
\usepackage[a4paper, margin=2cm]{geometry}
\usepackage{fontspec}
\usepackage{array}
\usepackage{booktabs}
\usepackage{graphicx}

\setmainfont{CMU Serif}

% ---------
\renewcommand{\thetable}{\arabic{table}}
\usepackage{longtable}
% ---------

\newcommand{\barchart}[1]{
  \includegraphics[width=0.25\textwidth]{{#1}}
}

\newcommand{\factortable}[3]{
  \noindent
  \begin{tabular}{@{}p{0.48\textwidth}|p{0.48\textwidth}@{}}
    \toprule
    \multicolumn{2}{@{}c@{}}{\textbf{Фактор #2}} \\
    \midrule
    \multicolumn{2}{@{}c@{}}{\barchart{#1}} \\
    \midrule
    #3 \\
    \bottomrule
  \end{tabular}
}

\newcommand{\visualsearch}[3]{
    \begin{center}
        \textbf{\large #1}
    \end{center}

    \vspace{1em}

    #2

    \vspace{1em}

    \includegraphics[width=0.8\textwidth]{#3}
}

\begin{document}

\begin{center}
\Large\textbf{Многофакторный личностный опросник Кеттелла (детский вариант)}
\end{center}

В ходе этого опросника ребенок отвечает на ряд вопросов, выбирая один из двух вариантов ответа.

Тест оценивает 14 факторов, отражающих характеристики некоторых качеств личности.

По результатам теста по каждому фактору подсчитываются баллы (от 1 до 10).

Низкие баллы (от 1 до 3) и высокие баллы (от 8 до 10) отражают два противоположных полюса каждого фактора.

Средние баллы, как правило, показывают баланс между двумя противоположными характеристиками.

\vspace{1cm}

<factors>

\vspace{2em}

\begin{center}
    \textbf{\large Прогрессивные матрицы Равена}
\end{center}

\vspace{1em}

\begin{tabular}{p{0.5\textwidth}|p{0.5\textwidth}}
    % \RaggedRight\small
    \sloppy\hyphenpenalty=10000 \exhyphenpenalty=10000
    В ходе теста ребенку показываются картинки, построенные на основе геометрических фигур, линий и узоров в соответствии с определенной логикой. 
    При этом, часть картинки отсутствует.
    Задача ребенка -- дополнить картинку, выбрав один из нескольких предложенных вариантов ответа.
    
    \vspace{0.5em}
    
    Данный тест является одним из общепринятых методов оценки \textbf{коэффициента интеллекта (IQ)}.
    
    \vspace{0.5em}
    
    Однако, учитывая специфику задания, тест оценивает скорее не общий, а пространственный интеллект. Это следует учитывать при интерпретации результатов.
    
    \vspace{0.5em}
    
    Баллы теста пересчитываются в IQ (число на рисунке справа) и затем сравниваются с результатами, полученными для большой группы детей того же возраста.
    Обычно нормой считаются значения IQ \textbf{от 85 до 115}.
    &
    \raisebox{-\totalheight}{<iq_image>}
    \\
\end{tabular}

\newpage

\begin{center}
    \textbf{\large Элементарные когнитивные функции}
\end{center}

\vspace{1em}

В исследовании оцениваются элементарные когнитивные функции -- \textbf{базовые способности мозга}, своего рода ``строительные блоки'', из которых мозг формирует стратегию выполнения более сложных задач.

Всего оценивается четыре функции: визуальный поиск, рабочая память, ментальная арифметика, способность комбинировать несколько функций.

Каждая функция оценивается с помощью своего теста, состоящего из некоторого числа отдельных заданий. 
В ходе выполнения каждого теста рассчитываются \textbf{4 характеристики}:

\begin{itemize}
  \item Процент ошибок -- доля неправильных ответов среди заданий текущего теста
  \item Время ответа -- среднее время ответа на задания текущего теста
  \item Усталость -- мера усталости, накопленной в ходе выполнения текущего теста
  \item Неустойчивость внимания --- мера, показывающая насколько сильно колеблется уровень внимания в ходе выполнения текущего теста
\end{itemize}

Из описания видно, что чем ниже каждая из характеристик, тем лучше.
 
На рисунках в виде столбчатых диаграмм показано, как результаты распределены среди участников.

Метка «ромбик» на рисунке показывает место результата ребенка в общей группе.

В подписи к рисунку приведена интерпретация результата ребенка:

\begin{itemize}
  \item Группа A -- результат выше, чем у 75\% участников
  \item Группа B -- результат на уровне 50\% участников
  \item Группа C -- результат ниже, чем у 75\% участников
\end{itemize}

Если по какой-либо когнитивной функции ребенок находится в \textbf{группе C} по \textbf{1 и более} характеристикам из 4, то данная когнитивная функция нуждается в дополнительном развитии.

\newpage

\visualsearch{Визуальный поиск}{
Ребенку показывается число, которое он должен запомнить.

\vspace{0.5em}

Затем показывается таблица чисел 5х5, в которой ребенок должен правильно и как можно быстрее найти ранее показанное число.
}{VisualSearch.png}

\vspace{0.5em}

\visualsearch{Рабочая память}{
Ребенку показывается несколько чисел, которые он должен запомнить.

\vspace{0.5em}

Затем после небольшой паузы показывается еще число, задача ребенка – вспомнить, было ли это число среди показанных ранее.
}{WorkingMemory.png}

\newpage

\visualsearch{Ментальная арифметика}{
Ребенку показывается математическое равенство типа A-B = C, равенство может быть как верным, так и неверным.

\vspace{0.5em}

Задача ребенка – посчитать в уме A-B и ответить, является это равенство верным или нет.
}{MentalArithmetic.png}

\vspace{0.5em}

\visualsearch{Комбинирование функций}{
Самая сложная задача, требующая комбинации трех других когнитивных функций.

\vspace{0.5em}

Ребенку показывается два числа – например, «76» и «2», а затем таблица чисел 5х5.

\vspace{0.5em}

Задача ребенка – найти в таблице сначала первое число (76), затем посчитать в уме разницу между первым и вторым числом (76-2 = 74), найти это число, затем посчитать разницу между новым числом и вторым числом (74-2 = 72), найти это число и т.д.
}{CombFunction.png}

\newpage

<RecomendationHeader>
<Recomendation>


\end{document}