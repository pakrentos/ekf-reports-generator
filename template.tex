\documentclass[10pt]{article}
\usepackage[a4paper, margin=2cm]{geometry}
\usepackage{fontspec}
\usepackage{array}
\usepackage{booktabs}
\usepackage{graphicx}

\setmainfont{CMU Serif}

\newcommand{\barchart}[1]{%
  \includegraphics[width=0.5\textwidth]{{#1}}%
}

\newcommand{\factortable}[3]{
  \noindent
  \begin{tabular}{@{}p{0.48\textwidth}|p{0.48\textwidth}@{}}
    \toprule
    \multicolumn{2}{@{}c@{}}{\textbf{Фактор #2}} \\
    \midrule
    \multicolumn{2}{@{}c@{}}{\barchart{#1}} \\
    \midrule
    #3 \\
    \bottomrule
  \end{tabular}
}

\newcommand{\visualsearch}[3]{
    \begin{center}
        \textbf{\large #1}
    \end{center}

    \vspace{1em}

    #2

    \vspace{1em}

    \includegraphics[width=\textwidth]{#3}
}

\begin{document}

\begin{center}
\Large\textbf{Многофакторный личностный опросник Кеттелла (детский вариант)}
\end{center}

В ходе этого опросника ребенок отвечает на ряд вопросов, выбирая один из двух вариантов ответа.

Тест оценивает 12 факторов, отражающих характеристики некоторых качеств личности.

По результатам теста по каждому фактору подсчитываются баллы (от 1 до 10).

Низкие баллы (от 1 до 3) и высокие баллы (от 8 до 10) отражают два противоположных полюса каждого фактора.

Средние баллы, как правило, показывают баланс между двумя противоположными характеристиками.

\vspace{1cm}

\factortable{<factor_a>}{A}{
Обособленность, отчужденность & Доброта, сердечность \\
Сухой, холодный, бесприсрастный & Эмоциональный, мягкосердечный \\
Скрытный, недоверительный & Открытый, доверчивый \\
Ригидный, упрямый & Гибкий, адаптивный \\
Тревожный, осторожный & Беззаботный, импульсивный \\
Сдержанный, конфликтный & Экспрессивный, доброжелательный \\
Замкнутый, недоверчивый, обособленный & Общительный, участливый
}

% \vspace{1cm}

\factortable{<factor_b>}{B}{
Низкие общие мыслительные способности & Высокие общие мыслительные способности \\
Не может решать абстрактные задачи & Может решать абстрактные задачи \\
Узкий спектр интеллектуальных интересов & Широкие интеллектуальные интересы \\
Медленно обучается & Быстро обучается \\
Медленно соображает & Быстро соображает \\
Не доводит дело до конца & Упорный, настойчивый \\
Объем знаний невелик & Большой объем знаний
}

% \vspace{1cm}

\factortable{<factor_c>}{C}{
Эмоциональная неустойчивость & Эмоциональная устойчивость \\
Легко расстраивается, тревожный & Спокойный \\
Имеет невротические симптомы & Свободный от невротических симптомов \\
Уклоняется от ответственности & Реалистичен в отношении к жизни \\
Невыдержанный, возбудимый, нетерпеливый & Умеет держать себя в руках \\
Неустойчив в интересах & Интересы постоянны \\
Неуверенный в себе, легко ранимый & Уверенный в себе, стабильный
}

% \vspace{1cm}

\factortable{<factor_d>}{D}{
Флегматичный, уравновешенный & Преобладает возбудимость \\
Спокойный, удовлетворенный & Беспокойный, нетерпеливый \\
Нервнивый, самокритичный, постоянный & Ревнивый, непостоянный \\
Тактичный & Эгоцентричный \\
Нетороплиый & Реактивный \\
Сдержанный & Высокое самомнение
}

% \vspace{1cm}

\factortable{<factor_e>}{E}{
Покорный, зависимый & Доминирующий, независимый \\
Неуверенный в себе, скр & Напористый, настойчивый \\омный
Робкий & Самоуверенный, хвастливый, тщеславный \\
Доброжелательный, послушный & Конфликтный, своенравный, мятежный \\
Тактичный & Бесцеремонный \\
Осторожный, уступчивый & Смелый, авантюристичный
}

% \vspace{1cm}

\factortable{<factor_f>}{F}{
Озабоченный, печальный, пессимистичный & Беспечный, жизнерадостный \\
Тревожный, уединенный & Общительный \\
Молчаливый, подозрительный & Разговорчивый, доверчивый \\
Вялый, апатичный & Бодрый, энергичный, импульсивный, гибкий, подвижный \\
Серьезный, осторожный & Легкомысленный, небрежный \\
Ответственный, благоразумный, рассудительный & Склонный к риску, храбрый, веселый
}

% \vspace{1cm}

\factortable{<factor_g>}{G}{
Недобросовестный, пренебрегающий обязанностями & Высокая совестливость \\
Безответственный, легкомысленный & Ответственный, обязательный \\
Беспринципный, незрелый & Моральный, зрелый \\
Расхлябанный, непостоянный & Настойчивый, упорный \\
Расслабленный & Требовательный к порядку, дисциплинированный \\
Потворствующий своим желаниям & Добросовестный, исполнительный
}

% \vspace{1cm}

\factortable{<factor_h>}{H}{
Робкий, застенчивый & Смелый, общительный \\
Чувствительный к опасности, осторожный & Беззаботный, не видит сигналов опасности \\
Сдержанный, озлобленный & Отзывчивый, дружелюбный \\
Смущается в присутствии противоположного пола & Оживляется в присутствии противоположного пола \\
Ограниченные интересы & Эмоциональные и художественные интересы \\
Уединенный & Социально-смелый, непринужденный, любит быть на виду
}

% \vspace{1cm}

\factortable{<factor_i>}{I}{
Суровый, жестокий, реалистичный & Мягкосердечный, нежный, сентиментальный \\
Практичный, «толстокожий» & Эстетически утонченный, ранимый \\
Не склонный к фантазиям & Склонный к фантазиям \\
Действует по логике & Действует по интуиции \\
Несколько черствый, независимый & Мягкий, добрый, зависимый \\
Самоудовлетворенный ответственный & Нуждающийся в помощи и внимании, легкомысленный \\
Не склонный к ипохондрии, эмоционально зрелый & Ипохондричный, нетерпеливый, непостоянный
}

% \vspace{1cm}

\factortable{<factor_o>}{O}{
Самоуверенный, уверенный адекватно & Склонный к чувству вины \\
Жизнерадостный, спокойный & Печальный, унылый, озабоченный \\
Нечувствительный к замечаниям и порицаниям & Чувствительный к замечаниям и порицаниям \\
Безмятежный & Ранимый, недооценивающий себя \\
Смелый, чувствующий себя в безопасности & Боязливый, не чувствующий себя в безопасности \\
Активный & Склонный к размышлениям \\
Расслабленный, оптимистичный & Напряженный, тревожный
}

% \vspace{1cm}

\factortable{<factor_q3>}{Q3}{
Импульсивность & Контроль желаний \\
Низкий самоконтроль поведения & Высокий самоконтроль поведения \\
Плохое понимание социальных нормативов & Хорошее понимание социальных нормативов
}

% \vspace{1cm}

\factortable{<factor_q4>}{Q4}{
Расслабленный & Напряженный \\
Спокойный, невозмутимый & Раздражительный
}

\vspace{2em}

\begin{center}
    \textbf{\large Прогрессивные матрицы Равена}
\end{center}

\vspace{1em}

\begin{tabular}{p{0.5\textwidth}|p{0.5\textwidth}}
    % \RaggedRight\small
    \sloppy\hyphenpenalty=10000 \exhyphenpenalty=10000
    В ходе теста ребенку нужно дополнить картинку, выбрав один из нескольких предложенных вариантов.
    
    \vspace{0.5em}
    
    Данный тест является одним из общепринятых методов оценки коэффициента интеллекта (IQ).
    
    \vspace{0.5em}
    
    Однако, учитывая специфику задания, данный тест оценивает скорее не общий, а пространственный интеллект. Это следует учитывать при интерпретации результатов.
    
    \vspace{0.5em}
    
    Баллы теста пересчитываются в IQ (число на рисунке справа) и затем сравниваются с результатами, полученными для большой группы детей того же возраста.
    &
    \raisebox{-\totalheight}{\includegraphics[width=\linewidth]{output_image.png}}
    \\
\end{tabular}

\vspace{1em}

\begin{center}
    \textbf{\large Элементарные когнитивные функции}
\end{center}

\vspace{1em}

В исследовании оцениваются элементарные когнитивные функции -- \textbf{базовые способности мозга}, своего рода ``строительные блоки'', из которых мозг формирует стратегию выполнения более сложных задач.

Всего оценивается четыре функции: визуальный поиск, рабочая память, ментальная арифметика, способность комбинировать несколько функций.

Каждая функция оценивается с помощью отдельного задания. В ходе выполнения задания рассчитывается \textbf{доля правильных ответов} (чем выше, тем лучше) и \textbf{среднее время ответа} (чем ниже, тем лучше).

На рисунках показано, как результаты распределены среди участников: каждый столбик показывает, сколько детей (высота столбика) имеют определенный результат (ширина столбика).

Метка «ромбик» на рисунке показывает место результата ребенка в общей группе.

В подписи к рисунку приведена интерпретация результата ребенка:

\begin{itemize}
  \item Группа A -- результат выше, чем у 66\% участников
  \item Группа B -- результат выше, чем у 33\% участников
  \item Группа C -- результат ниже, чем у 33\% участников
\end{itemize}

\vspace{0.5em}

\visualsearch{Визуальный поиск}{
Ребенку показывается число, которое он должен запомнить.

\vspace{0.5em}

Затем показывается таблица чисел 5х5, в которой ребенок должен правильно и как можно быстрее найти ранее показанное число.
}{VisualSearch.png}

\vspace{0.5em}

\visualsearch{Рабочая память}{
Ребенку показывается несколько чисел, которые он должен запомнить.

\vspace{0.5em}

Затем после небольшой паузы показывается еще число, задача ребенка – вспомнить, было ли это число среди показанных ранее.
}{WorkingMemory.png}

\vspace{0.5em}

\visualsearch{Ментальная арифметика}{
Ребенку показывается математическое равенство типа A-B = C, равенство может быть как верным, так и неверным.

\vspace{0.5em}

Задача ребенка – посчитать в уме A-B и ответить, является это равенство верным или нет.
}{MentalArithmetic.png}

\vspace{0.5em}

\visualsearch{Комбинирование функций}{
Самая сложная задача, требующая комбинации трех других когнитивных функций.

\vspace{0.5em}

Ребенку показывается два числа – например, «76» и «2», а затем таблица чисел 5х5.

\vspace{0.5em}

Задача ребенка – найти в таблице сначала первое число (76), затем посчитать в уме разницу между первым и вторым числом (76-2 = 74), найти это число, затем посчитать разницу между новым числом и вторым числом (74-2 = 72), найти это число и т.д.
}{CombFunction.png}

\end{document}